\chapter{Introduction}

Communication is essential to building relationships. A person who has challenges speaking will face a lifetime of roadblocks in building friendships, connecting with family, and meeting daily needs. In the United States, it is estimated that 1 out of every 68 children will be diagnosed with some level of Autism\footnote[1]{Maguire, Corinne. "Autism on the Rise: A Global Perspective." Harvard College Global Health Review. (retrieved October 7, 2016).}, and many of these children will face communication challenges or be rendered completely nonverbal, depriving these children of a voice. As this number continues to rise\footnote[2]{Christensen, Deborah L.. "Prevalence and Characteristics of Autism Spectrum Disorder Among Children Aged 8 Years ? Autism and Developmental Disabilities Monitoring Network, 11 Sites, United States, 2012." Centers for Disease Control and Prevention. (retrieved October 7, 2016).}, finding ways for everyone to clearly communicate is essential to enhancing the human experience.

Today, nonverbal children such as those diagnosed with autism or Down syndrome communicate using many methods including gestures, sign language, and picture symbols. One of the most popular methods of communication is a device that generates speech. These devices come in many different forms. Some are similar to keyboards on which the child can type out what they want to say, while others have a list of buttons with pre-programmed messages from which the child can choose. Both of these options have also been incorporated into touchscreen devices such as the Apple iPad, so they are easily portable. 

While the ability to type out any response gives a flexible voice to the children, it can be tedious to retype similar responses and frustrating for all involved due to the time it takes to construct responses using a keyboard. The solutions with pre-programmed messages solve this problem by speeding up communication, but they impede the expressiveness of the children by limiting their response options. These limitations do not allow subtleties in diction, syntax, and personal preference to be communicated, erasing the voice from the personality behind it.

Using Machine Learning principles, Artificial Intelligence and Natural Language Processing, we developed an application that makes conversations quick but also personalized, giving people their own voice. We propose a solution that listens to conversations and gives the child several quick options to speak. These options will be personalized to each child, so that they maintain their voice. Additionally, we will have the option to type out a response if the available quick options are not what the child wants to communicate. The typed answers will be used to help the system learn how the child responds, thereby improving future suggestions. We plan on testing this solution with nonverbal children at Hope Technology School in Palo Alto, California.

Our solution combines the benefits of both current solutions, while eliminating the problems. By having the system learn about the child on an individual level, this communication tool will allow children to share their voice with the world. The inclusion of the quick suggestions will drastically improve response time, easing the communication process both between children and between verbal adults, such as the teacher or parent, and the child. Communication is crucial to forming human connections. Our proposed solution allows for seamless, fluid communication, giving everyone a voice.
