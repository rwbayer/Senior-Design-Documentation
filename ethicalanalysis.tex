\chapter{Ethical Analysis}

Even though our project can seem benevolent, there are several ethically ambiguous scenarios that must be considered and weighed prior to making decisions.


First, at an organizational level, our team has a couple of ethical dilemmas to consider. One such dilemma is how to ensure workload is being divided evenly among teammates. Like many problems, there seems to be a simple solution, by making a list of the work to be done and divide it evenly by number of tasks. But this process and judgment gets trickier with added complications, such as certain tasks requiring different amounts of time or team members missing deadlines. There are an infinite amount of potential situations to assess, so a global solution must be developed that can be applied across multiple scenarios. A good solution template would begin with a group discussion to try to work out the issue. If not everyone is satisfied with the results, we should schedule a meeting with our advisor to ask for him or her to mediate the discussion, allowing the experienced advisor to offer advice or make decisions for the team if necessary. Though this mediation strategy is generic it can be applied across multiple scenarios, which makes it a good basis for resolving ethical issues.


Secondly, concerning the social aspects of our project, there are many more ethical scenarios to consider in order to ensure that our application maximizes social benefits while minimizing concerns. One question we must consider is if we, as the developers, should have the power to control someone's voice. Our application will try to predict what a user will typically say and how he or she would respond to a question. This gives us the capability to influence what options the user will choose in speaking. This feature can be potentially misused to influence and control someone's voice in an extreme case. Questions determining the extent of control this application need to be evaluated to determine what kind of control the application should have in determining how a user will say something or respond to a question.


Thirdly, we have to consider the ethical implications of how our product system will be implemented. Our application will need to save records of the user's conversations to remember the interests and voice patterns of the user. One situation to evaluate is the option to save that personal information locally on the user's device or on our servers in the cloud. Both options have their respective pros and cons, namely that the cloud storage will improve speed and battery usage, and local storage will improve security. These trade-offs will force us to evaluate the situation and have us weigh the priority of features to determine how secure we need to store the user's data. Likely, these priorities will be shifted as the application gets tested by users and they give feedback, so it will be important for our team to come back to this dilemma and re-evaluate the best storage option from time to time. The extent and implementation of features need to be evaluated and assessed in the development of our application with user safety in mind.


Conclusion Paragraph?

Notes from Google doc:
Bobby: I think we should include something in this section about developing a solution to a problem without the appropriate background (not dr.s or experienced in nonverbal communication techniques).
Davis: get lots of feedback and direction from the teacher and HTS. recognize that we are developers, not childhood development experts.

