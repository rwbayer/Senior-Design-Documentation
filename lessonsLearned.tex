\chapter{Conclusion}

Over the course of this project we climbed over many walls to develop a consistent and finished product. This section contains results of how successful our design was, the lessons we learned along the way that helped develop our understanding and success in tackling large projects, and finally the future work we envision to build upon the project and create en even more robust and thorough solution.

\section{Results}
Our implemented design proved to be more effective at enabling quick and personalized conversation than other top solutions in the field. One way that our design improved the personalization of conversation was through our eight category response prediction algorithm. This algorithm sends the input statement to Google Cloud's Natural Language Processing API and receives a syntax and semantic breakdown of the statement. From this response, our algorithm determines which category of response is requested. For example, the question, "Where do you want to go for dinner?" will expect a "location" type response, and on the other hand the question "Who do you want to pick you up from school today?" will expect a "person" type response. Our system's ability to break down expected responses into eight different categories speeds up the user's ability to choose the right response they want to say and does not require them to make trade-offs in selecting an imperfect response to save time.

The next metric we used to classify the results of our system was a common metric in Human Computer Interaction (HCI) called the Number of Clicks metric. This simple metric measures the number of interactions a user must make with the app to perform the desired task. This is an important metric because speed is a huge factor in the success of a conversation, and therefore important to the success of our system. After our first meeting with our industry expert at the Hope Technology School, Christian Trifforiot confirmed that an average "click time" for a typical child with a nonverbal disability would take about six seconds, and that our application on average takes an average of 1.5 clicks to perform a desired operation \footnote{Trifforiot, Christian. (10 March 2017). Personal Interview}. Meaning that our application would take about nine seconds on average to formulate a response. Trifforiot also claimed that a leading competitor application called Proloquo2go \footnote{AssistiveWare. "Proloquo2go". Mobile Application. https://itunes.apple.com/us/app/ proloquo2go-symbol-based-aac/id308368164?mt=8} takes 7 clicks on average to perform a desired operation, meaning it can take up to 42 seconds to form a response. Our application is able to cut down response time to more than \(\frac{1}{3}\) of the response time by a leading application. This leads to a better user experience and improved conversation for the user and their converser.

\section{Lessons Learned}
Through much research, advice, and trial-and-error, we learned a lot of lessons that allowed us to deliver a completed project. We will continue to follow these principles and learn more as we take on any other projects and join other teams in the future.
\begin{enumerate}
	\item \textbf{Be wary of dependencies.} The more outside tools that your project relies on, the more points of failure are induced. Be careful  of using too many tools that are outside of your control, and be prepared for them to suddenly change or not work.
	\item \textbf{Multiply estimated time by 5.} In most circumstances, whatever time you estimate a task to take, it will take longer. Planning for unexpected development time will provide for a more accurate time assessment. 
	\item \textbf{Seek feedback early and often.} To avoid unexpected changes of requirements and unnecessary implementations of features that the customer may not want, maintain continuous contact with the customer. The earlier along in the development stages you can get feedback, the easier it is to iterate on those changes.
	\item \textbf{Build it up piece by piece.} Incremental design will allow for easy feature integration and natural divisions for unit testing. Building piece by piece will improve code understanding and reduce the number of unexpected errors.
\end{enumerate}

\section{Future Work}
Though our project demonstrated how using Natural Language Processing could be an invaluable tool in the application of predicting responses for people with nonverbal disabilities, there are many ways we see that our project could be improved. The following list details all the improvements we want to see developed to continue to help those 
\begin{enumerate}
	\item \textbf{Settings for voice options, number of displayed presets, and custom images in categories.}  There are many customization options we want to add to improve the app. We want to incorporate voice options for the user to select a male or female voice, because it is important for the user to really be able to identify the computer's voice as their own. For example, if a boy is using our application he may feel uncomfortable speaking with a female's voice, and vice versa. Next, we want to have a setting to adjust the number of displayed picture cards on the screen to raise or lower the complexity of decision-making for the user, depending on their age and level of ability. Finally, we want to provide an option to incorporate a user's own images in the picture card responses. For example, if a user gets prompted with the "Mom" response, it can be much more helpful for some children with mental disabilities to process a picture of their actual mother rather than a cartoon picture of a woman. Adding these options can significantly improve user experience and user retention for our application.
	\item \textbf{Improve question to answer mapping.} Our application talks to our server and receives feedback to predict what type of response is expected based on an input statement. Due to the limitations of Natural Language Processing, there are some cases where we cannot correctly predict what the user would actually say in response to a question. For example, if the input question reads, "What do you think of the Mona Lisa?", our application interprets Mona Lisa as a person, rather than a work of art. So, the user is presented with a person category rather than a work-of-art category. This is a small example that highlights the complexity of Natural Language Processing, but also shows us a road map for what developers and linguists can do to improve the tool and make it more universal and helpful for any situation.
	\item \textbf{Form sentences with suggested responses.} The current state of our application allows the user to choose a phrase or word in response to an input statement. We would like to expand this response into a completed sentence to improve conversation flow, truly giving the user their own uninhibited voice. For example, when the user is given the input statement of "What would you like to play with for recess?", the user can select from a list of items, and may say and answer like "basketball". Instead of giving a one-word answer, we would love to expand the response to something more like, "I would like to play basketball today," to make the conversation more realistic. This could improve the sociability of the user, teaching young children how to formulate sentences, and make it easier for the converser to understand the user's response.
	\item \textbf{Expand to other platforms.} Finally, we see lots of potential for how this system can be incorporated and integrated onto other platforms, especially wearable devices. The first step would be to make this available as an Android app so it is available on a lot more devices, especially to the majority of devices in third-world countries where Android devices are much more prevalent than iOS devices due to cost. Next, to allow this to be used on a smart watch or smart glasses like a pair of google glasses would allow the user the ability to converse through their device without needing to carry around an iPad everywhere they go. This would greatly increase the confidence and mobility of the user, also creating a more enjoyable user experience. The less invasive the technology is to the user's life and functions, the easier the system is able to integrate with anyone's lifestyle and circumstances.
\end{enumerate}
